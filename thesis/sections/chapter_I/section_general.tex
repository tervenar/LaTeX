% declaration of document class
\documentclass[12pt, class = article, crop = false, a4paper, twoside]{standalone}

% external packages for document style formatting
\usepackage[left = 2.5cm, right = 2.5cm, top = 2.5cm, bottom = 2.5cm]{geometry}

% input style file
% external packages for text formatting
\usepackage[shortlabels]{enumitem}
\usepackage[utf8]{inputenc}
\usepackage[english]{babel}
\usepackage{csquotes}
\usepackage{fancyref}

% text formatting
\setlist[enumerate]{itemsep = 0mm}
\setlength{\parindent}{0cm}
\usepackage{setspace}
%\onehalfspacing
%\singlespacing

% external packages for maths environments
\usepackage{amssymb, amsthm}
\usepackage{mathrsfs}
\usepackage{mathtools}

% external packages for figures
\usepackage[pdftex]{graphicx} % demo
\usepackage{caption}
\usepackage{subcaption}

% external packages for standalone and cross-referencing
\usepackage[subpreambles = false]{standalone}
\usepackage{import}

% bibliography style formatting
% bibliography style formatting
\usepackage[
    backend = biber,
    natbib = true,
    style = alphabetic,
    sorting = nyt,
    doi = false,
    isbn = false,
    url = false,
    eprint = false
]{biblatex}

% A 
\addbibresource{../../bibliography/ARFI22.bib}
\addbibresource{bibliography/ARWI03.bib}
\addbibresource{bibliography/ARWI11.bib}
\addbibresource{bibliography/ARWI76.bib}
\addbibresource{bibliography/AVAN73.bib}

% C 
\addbibresource{bibliography/CHDO+19.bib}
\addbibresource{bibliography/CHER08.bib}

% D
\addbibresource{bibliography/DAKE15.bib}
\addbibresource{bibliography/DOAD21.bib}
\addbibresource{bibliography/DODU23.bib}
\addbibresource{../../bibliography/DOMA17.bib}
\addbibresource{bibliography/DOSH23.bib}
\addbibresource{bibliography/DRMC05.bib}
\addbibresource{bibliography/DUMA22.bib}

% G
\addbibresource{bibliography/GOSE14.bib}

% K 
\addbibresource{bibliography/KAVE83.bib}
\addbibresource{bibliography/KEHA59.bib}
\addbibresource{bibliography/KEHA63.bib}

% L
\addbibresource{bibliography/LAST91.bib}

% M 
\addbibresource{bibliography/MAGR66.bib}
\addbibresource{bibliography/MUCA19.bib}

% S 
\addbibresource{bibliography/SAST78.bib}
\addbibresource{bibliography/SAST97.bib}
\addbibresource{bibliography/SEEU81.bib}
\addbibresource{bibliography/STCH66.bib}

% W 
\addbibresource{bibliography/WOWO00.bib}
\addbibresource{bibliography/WOWO09.bib}
\addbibresource{bibliography/WOWO21.bib}

% Y 
\addbibresource{bibliography/YAAR16.bib}
\addbibresource{bibliography/YAAR24.bib}

% paths for images
\graphicspath{{images/figures}}

% declaration of environments and commands
% environment for theorems
\newtheorem{theorem}{Theorem}[section]
\newtheorem{corollary}{Corollary}[theorem]

% environment for lemmas
\newtheorem{lemma}{Lemma}[section]

% environment for propositions
\newtheorem{proposition}{Proposition}[section]

% environments for non-provable
\theoremstyle{definition}
\newtheorem{definition}{Definition}[section]
\newtheorem{example}{Example}[section]
\theoremstyle{remark}
\newtheorem*{remark}{Remark}
\newtheorem*{notation}{Notation}


% define maths commands

% define command
\newcommand{\mbinom}[2]{\left|\begin{matrix}#1 \\ #2\end{matrix}\right|}
\newcommand{\smbinom}[2]{\left|\begin{smallmatrix}#1 \\ #2\end{smallmatrix}\right|}
\newcommand{\abs}[1]{\left| #1 \right|}

% define notation
\newcommand{\B}{\mathbb{B}}
\newcommand{\C}{\mathbb{C}}
\newcommand{\F}{\mathbb{F}}
\newcommand{\G}{\Gamma}
\newcommand{\K}{\mathbb{K}}
\newcommand{\N}{\mathbb{N}}
\newcommand{\Q}{\mathbb{Q}}
\newcommand{\R}{\mathbb{R}}
\newcommand{\T}{\mathbb{T}}
\newcommand{\Z}{\mathbb{Z}}
\newcommand{\const}{\operatorname{const.}}
\newcommand{\Span}{\operatorname{span}}
\newcommand{\supp}{\operatorname{supp}}
\newcommand{\Mod}{\operatorname{mod}}

% define operators
\DeclareMathOperator\Ad{Ad}
\DeclareMathOperator\Id{Id}
\DeclareMathOperator\ev{ev}
\DeclareMathOperator\Gr{Gr}
\DeclareMathOperator\ST{ST}
\DeclareMathOperator\LL{LL}
\DeclareMathOperator\dom{dom}
\DeclareMathOperator\Alg{Alg}
\DeclareMathOperator\env{env}
\DeclareMathOperator\sign{sign}
\renewcommand{\P}{\mathbb{P}} 
\renewcommand{\Pr}{\operatorname{P}} 
\renewcommand{\Re}{\operatorname{Re}} 
\renewcommand{\Im}{\operatorname{Im}}
\renewcommand{\dim}{\operatorname{dim}}


% helper for cross-referencing
\makeatletter
\newcommand*{\addFileDependency}[1]{
  \typeout{(#1)}
  \@addtofilelist{#1}
  \IfFileExists{#1}{}{\typeout{No file #1.}}
}
\makeatother

\newcommand*{\myexternaldocument}[1]{
    \externaldocument{#1}
    \addFileDependency{#1.tex}
    \addFileDependency{#1.aux}
}
\usepackage{xr}

% list of sections for referencing
%\myexternaldocument{section_splines}

\begin{document}

% section declaration
\section{Ideals and representations}
\label{sec:definitions}

% main body
\subsection{Space-time Martin boundary}

We now provide definitions and notation for the space-time Martin boundary as in [REF].

\begin{definition}[Space-time graph]

    Let $\Gamma$ be a countable discrete group, and $\mu$ be an admissible probability measure on $\Gamma$. Then for $z\in\Gamma$, the set of vertices 
    \begin{equation*}
        \ST_{z} = \ST_{z}(\Gamma, \mu) := \{(y, m)\in\Gamma\times\Z_{+} \,:\, P^{m}(y, z) > 0\},
    \end{equation*}
    where we add an edge between $(x, n), (y, m)\in\ST_{z}$ if $m = n + 1$ and $P(x, y) > 0$,
    is called the \textit{space-time graph} of the random walk $(\Gamma, \mu)$.
\end{definition}

\begin{definition}[Space-time Martin chain]
\label{def:chain:space-time}

    Let $\Gamma$ be a countable discrete group, and $\mu$ be an admissible probability measure on $\Gamma$. Then the Markov chain $(Y_{n})_{n\in\N}$ on $\ST_{e}$ with the transition probabilities given by
    \begin{equation*}
        \P[Y_{n + 1} = (y, m + 1) \,|\, Y_{n} = (x, m) ] = P(x, y),
    \end{equation*}
    is called the \textit{space-time Martin chain}.
\end{definition}

\begin{remark}
    Note that for $\ST_{z}$, the transition probabilities take the form
    \begin{equation*}
        \P[Y_{n + 1} = (y, m + 1) \,|\, Y_{n} = (x, m) ] = ?,
    \end{equation*}
\end{remark}

\begin{remark}
    Transience and reducibility
\end{remark}

The following proposition serves as a definition for a space-time Martin kernel. 
\begin{proposition}[REF]

    Let $\Gamma$ be a countable discrete group, and $\mu$ be an admissible probability measure on $\Gamma$. Then the corresponding space-time Martin kernel is given by
    \begin{equation*}
        \begin{split}
            K_{ST_{e}}\colon\ST_{e}\times\ST_{e} & \to (0, \infty), \\
            ((x, m), (y, n)) & \mapsto \frac{P^{n - m}(x, y)}{P^{n}(e, y)}.
        \end{split}
    \end{equation*}
\end{proposition}

\begin{remark}

    Using the above remark, we see that for an arbitrary $z\in\Gamma$, the space-time Martin kernel is given by
    \begin{equation*}
        \begin{split}
            K_{\ST_{z}}\colon\ST_{z}\times\ST_{z} & \to (0, \infty), \\
            ((x, m), (y, n)) & \mapsto ?.
        \end{split}
    \end{equation*}
\end{remark}

\begin{definition}[Space-time Martin compactification]

    Let $\Gamma$ be a countable discrete group, and $\mu$ be an admissible probability measure on $\Gamma$. Then the $1$-Martin compactification of the space-time Markov chain associated with $(\Gamma, \mu)$ is called the \textit{space-time Martin compactification} of $\Gamma$ and is denoted by $\Delta_{\ST}\Gamma$. We denote the corresponding space-time Martin boundary by $\partial_{\ST}\Gamma$.
\end{definition}

\begin{proposition}[REF]

    Let $\Gamma$ be a countable discrete group, $\mu$ be an admissible probability measure on $\Gamma$. Then a sequence $(y_{k}, n_{k})_{k\in\N}$ in $\ST_{e}$ converges to a point $\xi\in\partial_{ST}\Gamma$ if for all $(x, m)\in\ST_{e}$,
    \begin{equation*}
        \frac{P^{n - m}(x, y)}{P^{n}(e, y)}\xrightarrow{k\to\infty}h_{\xi}(x, m)
    \end{equation*}
    for some function $h_{\xi}\colon\ST_{e}\to\R_{+}$.
\end{proposition}

\subsection{Toeplitz algebra and representations}
\begin{definition}[Fock space]

    Let $\Gamma$ be a countable discrete group, and $\mu$ be an admissible probability measure on $\Gamma$. Then, the Hilbert space
    \begin{equation*}
        \mathcal{H} = \mathcal{H}(\Gamma, \mu) := \bigoplus_{z\in\Gamma}\ell^{2}(\ST_{z}(\Gamma, \mu))
    \end{equation*}
    is called the \textit{Fock space} of the random walk $(\Gamma, \mu)$. For each $z\in\Gamma$, we let $\{e^{(m)}_{x, z}\}_{(x, m)\in\ST_{z}}$ denote the orthonormal basis of the space
    \begin{equation*}
        \mathcal{H}_{z} = \mathcal{H}_{z}(\Gamma, \mu) := \ell^{2}(\ST_{z}(\Gamma, \mu)).
    \end{equation*}
\end{definition}

\begin{remark}

    Notice that trough the identification $\oplus_{m\in\N}\ell^{2}(\ST^{(m)}_{z}) = \ell^{2}(\sqcup_{m\in\N}\ST^{(m)}_{z})$
    we obtain the following natural decomposition
    $\oplus_{z\in\Gamma}\mathcal{H}_{z} = \oplus_{z\in\Gamma}\oplus_{m\in\N}\mathcal{H}^{(m)}_{z}$,
    where we denote $\mathcal{H}^{(m)}_{z} := \ell^{2}(\ST^{(m)}_{z})$.
\end{remark}

\begin{remark}

    Note that the map $\ST_{z}\to\ST_{g^{-1}z}, (y, m)\mapsto(g^{-1}y, m)$ induces a canonical action of the group $\Gamma\curvearrowright\mathcal{H}(\Gamma, \mu)$ via the unitaries 
    \begin{equation*}
        \begin{split}
            U_{g}\colon\mathcal{H}(\Gamma, \mu) & \to \mathcal{H}(\Gamma, \mu), \\
            e^{(m)}_{x, z} & \mapsto e^{(m)}_{g^{-1}x, g^{-1}z}.
        \end{split}
    \end{equation*}
    These maps are well-defined since $U_{g}(\mathcal{H}_{z}) = \mathcal{H}_{g^{-1}z}$, and each of them sends an orthonormal basis to an orthonormal basis.
\end{remark}

\begin{remark}

    The space \dots in \cite{DOMA17} is isometrically isomorphic to the Fock space above. We change the notation to allude to the space-time Markov chain defined on the space-time graph.
\end{remark}

\begin{notation}

    For $n, m\in\N$ and $(x, y)\in\Gr(P)$, denote the bounded operators
    \begin{equation*}
        \begin{split}
            S^{(n)}_{x, y}\colon \mathcal{H}(\Gamma, \mu) & \to\mathcal{H}(\Gamma, \mu), \\
            e^{(m)}_{y', z} & \mapsto\delta_{y, y'}\sqrt{\frac{P^{n}(x, y)P^{m}(y, z)}{P^{n + m}(x, z)}}e^{(n + m)}_{x, z},
        \end{split}
    \end{equation*}
    where the adjoints are given by
    \begin{equation*}
        (S^{(n)}_{x, y})^{\ast}(e^{(n + m)}_{x', z}) = \delta_{x, x'}\sqrt{\frac{P^{m}(x, y)P^{n}(y, z)}{P^{m + n}(x, z)}}e^{(n)}_{y, z}.
    \end{equation*}
    Moreover, we have that $S^{(n)}_{x, y}(\mathcal{H}^{(m)}_{z})\subset \mathcal{H}^{(n + m)}_{z}$.
\end{notation}

\begin{definition}[Toeplitz algebra, tensor algebra]

    Let $\Gamma$ be a countable discrete group, and $\mu$ be an admissible probability measure on $\Gamma$. Then, 
    \begin{enumerate}[(i)]
        \item the $C^{\ast}$-algebra
            \begin{equation}
                \mathcal{T}(\Gamma, \mu) := C^{*}(S^{(n)}_{x, y} \,:\, (x, y)\in\Gr(P^{n}), n\in\N) \subset \mathbb{B}(\mathcal{H}(\Gamma, \mu))
            \end{equation}
            is called the \textit{Toeplitz algebra} associated with the random walk $(\Gamma, \mu)$;
        \item the operator algebra
            \begin{equation}
                \mathcal{T}^{+}(\Gamma, \mu) := \overline{\Alg}^{\|\cdot\|}(S^{(n)}_{x, y} \,:\, (x, y)\in\Gr(P^{n}), n\in\N) \subset \mathbb{B}(\mathcal{H}(\Gamma, \mu))
            \end{equation}
            is called the \textit{tensor algebra} associated with the random walk $(\Gamma, \mu)$.
    \end{enumerate}
\end{definition}

\begin{definition}[Cuntz algebra]

    Let $\Gamma$ be a countable discrete group, and $\mu$ be an admissible probability measure on $\Gamma$. Then, the space 
    \begin{equation}
        \mathcal{J} = \mathcal{J}(\Gamma, \mu) := \mathcal{T}(\Gamma, \mu)\cap\prod_{z\in \Gamma}\K(\mathcal{H}_{z}(\Gamma, \mu))
    \end{equation}
    is called the \textit{Cuntz ideal} of $\mathcal{T}(\Gamma, \mu)$, and the quotient $C^{\ast}$-algebra
    \begin{equation}
        \mathcal{O} = \mathcal{O}(\Gamma, \mu) := \mathcal{T}(\Gamma, \mu)/\mathcal{J}(\Gamma, \mu)
    \end{equation}
    is called the \textit{Cuntz algebra} associated with the random walk $(\Gamma, \mu)$.
\end{definition}

\begin{definition}[Cuntz-Pimsner-Viselter algebra]

    Let $\Gamma$ be a countable discrete group, and $\mu$ be an admissible probability measure on $\Gamma$. Then, the space 
    \begin{equation}
        \mathcal{I} = \mathcal{I}(\Gamma, \mu) := \bigoplus_{z\in \Gamma}\K(\mathcal{H}_{z}(\Gamma, \mu))
    \end{equation}
    is called the \textit{Cuntz-Pimsner-Viselter ideal} of $\mathcal{T}(\Gamma, \mu)$, and the quotient $C^{\ast}$-algebra
    \begin{equation}
        \mathcal{V} = \mathcal{V}(\Gamma, \mu) := \mathcal{T}(\Gamma, \mu)/\mathcal{I}(\Gamma, \mu)
    \end{equation}
    is called the \textit{Cuntz-Pimsner-Viselter algebra} associated with the random walk $(\Gamma, \mu)$.
\end{definition}

\begin{notation}

    For $z\in\Gamma$, we let $\pi_{z}$ denote the representations of $\mathcal{T}(\Gamma, \mu)$ given by
    \begin{equation}
        \begin{split}
            \pi_{z}\colon \mathcal{T}(\Gamma, \mu) & \to \mathbb{B}(\mathcal{H}_{z}(\Gamma, \mu)), \\
            T & \mapsto T|_{\mathcal{H}_{z}},
        \end{split}
    \end{equation}
    and for $n\in\N$, we let $\pi^{(n)}_{z}$ denote the representations of $n$-th amplification $M_{n}(\mathcal{T}(\Gamma, \mu))$ given by
    \begin{equation}
        \begin{split}
            \pi^{(n)}_{z}\colon M_{n}(\mathcal{T}(\Gamma, \mu)) & \to \mathbb{B}(\mathcal{H}_{z}(\Gamma, \mu)^{^{\oplus n}}), \\
            [T_{ij}]^{n}_{i, j = 1} & \mapsto [T_{ij}|_{\mathcal{H}_{z}}]^{n}_{i, j = 1}.
        \end{split}
    \end{equation}
\end{notation}

\begin{remark}

    We can extend an action $\Gamma\curvearrowright\mathcal{H}(\Gamma, \mu)$ via unitaries to a natural action $\Gamma\curvearrowright\mathcal{T}(\Gamma, \mu)$ via isometric $\ast$-isomorphisms 
    \begin{equation*}
        \begin{split}
            \Ad_{g}\colon\mathcal{T}(\Gamma, \mu) & \to \mathcal{T}(\Gamma, \mu), \\
            S^{(n)}_{x, y} & \mapsto U_{g}S^{(n)}_{x, y}U^{-1}_{g}.
        \end{split}
    \end{equation*}
    These maps are well-defined since for $g\in\Gamma$, $n, m\in\N$, and $(x, y)\in\Gr(P)$, we get
    \begin{equation*}
        \begin{split}
            (U_{g}S^{(n)}_{x, y}U^{-1}_{g})(e^{(m)}_{y', z}) & = U_{g}S^{(n)}_{x, y}(e^{(m)}_{gy', gz}) \\
            & = U_{g}\left(\delta_{y, gy'}\sqrt{\frac{P^{n}(x, y)P^{m}(y, gz)}{P^{n + m}(x, gz)}}e^{(n + m)}_{x, gz}\right) \\
            & = \delta_{y, gy'}\sqrt{\frac{P^{n}(x, y)P^{m}(y, gz)}{P^{n + m}(x, gz)}}e^{(n + m)}_{g^{-1}x, z} \\
            & = \delta_{g^{-1}y, y'}\sqrt{\frac{P^{n}(g^{-1}x, g^{-1}y)P^{m}(g^{-1}y, z)}{P^{n + m}(g^{-1}x, z)}}e^{(n + m)}_{g^{-1}x, z} \\
            & = S^{(n)}_{g^{-1}x, g^{-1}y}(e^{(m)}_{y', z}).
        \end{split}
    \end{equation*}
    This action leaves the tensor algebra invariant, and, moreover, is compatible with the representations above in the sense that $\pi_{z}\circ\Ad_{g} = \pi_{gz}$
    as for $g\in\Gamma$, $n\in\N$ and $(x, y)\in\Gr(P)$, we obtain
    \begin{equation*}
        \pi_{z}\circ\Ad_{g}(S^{(n)}_{x, y}) = S^{(n)}_{g^{-1}x, g^{-1}y}|_{\mathcal{H}_{z}} = S^{(n)}_{x, y}|_{\mathcal{H}_{gz}} = \pi_{gz}(S^{(n)}_{x, y}).
    \end{equation*}
    Note that the representations are not unitarily equivalent for different elements of the group, so $\Gamma$ can not act via unitaries.
\end{remark}

From \cite[Proposition 4.4]{DOAD21}, we see that $\mathcal{I}(\Gamma, \mu)$ is an ideal in the Toeplitz algebra $\mathcal{T}(\Gamma, \mu)$ whenever $\mu$ is finitely supported. Similar to \cite[Section 3]{DOMA17}, using the following short exact sequence 
\begin{equation*}
    0\to \mathcal{I}(\Gamma, \mu) \to \mathcal{T}(\Gamma, \mu) \to \mathcal{V}(\Gamma, \mu) \to 0
\end{equation*}
and the discussion preceding \cite[Theorem 1.3.4]{ARWI76}, we see that given any representation $\rho$, it decomposes uniquely into a central direct sum $\rho = \rho_{\mathcal{I}}\oplus\rho_{\mathcal{V}}$ of representations, where $\rho_{\mathcal{I}}$ is the unique extension of $\rho|_{\mathcal{I}(\Gamma, \mu)}$ to $\mathcal{T}(\Gamma, \mu)$, and $\rho_{\mathcal{V}}$ annihilates $\mathcal{I}(\Gamma, \mu)$. Note that for an irreducible representation, one of the summands is trivial.

\end{document}

