% declaration of document class
\documentclass[12pt, class = article, crop = false, a4paper, twoside]{standalone}

% external packages for document style formatting
\usepackage[left = 2.5cm, right = 2.5cm, top = 2.5cm, bottom = 2.5cm]{geometry}

% input style file
% external packages for text formatting
\usepackage[shortlabels]{enumitem}
\usepackage[utf8]{inputenc}
\usepackage[english]{babel}
\usepackage{csquotes}
\usepackage{fancyref}

% text formatting
\setlist[enumerate]{itemsep = 0mm}
\setlength{\parindent}{0cm}
\usepackage{setspace}
%\onehalfspacing
%\singlespacing

% external packages for maths environments
\usepackage{amssymb, amsthm}
\usepackage{mathrsfs}
\usepackage{mathtools}

% external packages for figures
\usepackage[pdftex]{graphicx} % demo
\usepackage{caption}
\usepackage{subcaption}

% external packages for standalone and cross-referencing
\usepackage[subpreambles = false]{standalone}
\usepackage{import}

% bibliography style formatting
% bibliography style formatting
\usepackage[
    backend = biber,
    natbib = true,
    style = alphabetic,
    sorting = nyt,
    doi = false,
    isbn = false,
    url = false,
    eprint = false
]{biblatex}

% A 
\addbibresource{../../bibliography/ARFI22.bib}
\addbibresource{bibliography/ARWI03.bib}
\addbibresource{bibliography/ARWI11.bib}
\addbibresource{bibliography/ARWI76.bib}
\addbibresource{bibliography/AVAN73.bib}

% C 
\addbibresource{bibliography/CHDO+19.bib}
\addbibresource{bibliography/CHER08.bib}

% D
\addbibresource{bibliography/DAKE15.bib}
\addbibresource{bibliography/DOAD21.bib}
\addbibresource{bibliography/DODU23.bib}
\addbibresource{../../bibliography/DOMA17.bib}
\addbibresource{bibliography/DOSH23.bib}
\addbibresource{bibliography/DRMC05.bib}
\addbibresource{bibliography/DUMA22.bib}

% G
\addbibresource{bibliography/GOSE14.bib}

% K 
\addbibresource{bibliography/KAVE83.bib}
\addbibresource{bibliography/KEHA59.bib}
\addbibresource{bibliography/KEHA63.bib}

% L
\addbibresource{bibliography/LAST91.bib}

% M 
\addbibresource{bibliography/MAGR66.bib}
\addbibresource{bibliography/MUCA19.bib}

% S 
\addbibresource{bibliography/SAST78.bib}
\addbibresource{bibliography/SAST97.bib}
\addbibresource{bibliography/SEEU81.bib}
\addbibresource{bibliography/STCH66.bib}

% W 
\addbibresource{bibliography/WOWO00.bib}
\addbibresource{bibliography/WOWO09.bib}
\addbibresource{bibliography/WOWO21.bib}

% Y 
\addbibresource{bibliography/YAAR16.bib}
\addbibresource{bibliography/YAAR24.bib}

% paths for images
\graphicspath{{images/figures}}

% declaration of environments and commands
% environment for theorems
\newtheorem{theorem}{Theorem}[section]
\newtheorem{corollary}{Corollary}[theorem]

% environment for lemmas
\newtheorem{lemma}{Lemma}[section]

% environment for propositions
\newtheorem{proposition}{Proposition}[section]

% environments for non-provable
\theoremstyle{definition}
\newtheorem{definition}{Definition}[section]
\newtheorem{example}{Example}[section]
\theoremstyle{remark}
\newtheorem*{remark}{Remark}
\newtheorem*{notation}{Notation}


% define maths commands

% define command
\newcommand{\mbinom}[2]{\left|\begin{matrix}#1 \\ #2\end{matrix}\right|}
\newcommand{\smbinom}[2]{\left|\begin{smallmatrix}#1 \\ #2\end{smallmatrix}\right|}
\newcommand{\abs}[1]{\left| #1 \right|}

% define notation
\newcommand{\B}{\mathbb{B}}
\newcommand{\C}{\mathbb{C}}
\newcommand{\F}{\mathbb{F}}
\newcommand{\G}{\Gamma}
\newcommand{\K}{\mathbb{K}}
\newcommand{\N}{\mathbb{N}}
\newcommand{\Q}{\mathbb{Q}}
\newcommand{\R}{\mathbb{R}}
\newcommand{\T}{\mathbb{T}}
\newcommand{\Z}{\mathbb{Z}}
\newcommand{\const}{\operatorname{const.}}
\newcommand{\Span}{\operatorname{span}}
\newcommand{\supp}{\operatorname{supp}}
\newcommand{\Mod}{\operatorname{mod}}

% define operators
\DeclareMathOperator\Ad{Ad}
\DeclareMathOperator\Id{Id}
\DeclareMathOperator\ev{ev}
\DeclareMathOperator\Gr{Gr}
\DeclareMathOperator\ST{ST}
\DeclareMathOperator\LL{LL}
\DeclareMathOperator\dom{dom}
\DeclareMathOperator\Alg{Alg}
\DeclareMathOperator\env{env}
\DeclareMathOperator\sign{sign}
\renewcommand{\P}{\mathbb{P}} 
\renewcommand{\Pr}{\operatorname{P}} 
\renewcommand{\Re}{\operatorname{Re}} 
\renewcommand{\Im}{\operatorname{Im}}
\renewcommand{\dim}{\operatorname{dim}}


% helper for cross-referencing
\makeatletter
\newcommand*{\addFileDependency}[1]{
  \typeout{(#1)}
  \@addtofilelist{#1}
  \IfFileExists{#1}{}{\typeout{No file #1.}}
}
\makeatother

\newcommand*{\myexternaldocument}[1]{
    \externaldocument{#1}
    \addFileDependency{#1.tex}
    \addFileDependency{#1.aux}
}
\usepackage{xr}

% list of sections for referencing
%\myexternaldocument{section_splines}

\begin{document}

% section declaration
\section{Simple random walk on the free group}
\label{sec:free}

% main body
We first introduce a minor technical lemma similar to \cite[Lemma 3.10]{DOMA17}.
\begin{proposition}
\label{prop:probability:inequalities}

    Let $\Gamma$ be a countable discrete group, and $\mu$ be an admissible aperiodic probability measure on $\Gamma$. Then
    \begin{enumerate}[(i)]
        \item there exists $n_{0}\in\N$ such that for any $n\geq n_{0}$ and any $x, y\in\Gamma$, we get 
        \begin{equation*}
            0 < P^{n}(x, y) < 1;
        \end{equation*}
        \item for $x, y\in\Gamma$ with $x\neq y$, there exists $n\in\N$ such that $0 < P^{n}(x, x) < 1$ and for all $m\in\N$ with $(y, m)\in\ST_{x}$, we have 
        \begin{equation*}
            P^{n}(x, x)P^{m}(x, y) < P^{n + m}(x, y).
        \end{equation*}
    \end{enumerate}
\end{proposition}

\begin{proof}

    
\end{proof}

\begin{notation}

    For $d > 0$, let $\F_{d}$ be the free group of $d$ generators $\{a_{1}, a_{2}, \dots, a_{d}\}$, and $\partial \F_{d}$ be its Gromov boundary, i.e. the space of infinite reduced words. For $(n, x)\in\ST_{e}$, define $\xi(n, x) = (\xi_{1}, \xi_{2}, \dots, \xi_{d})\in\R^{d}$ to be the vector of non-negative real numbers such that $n\xi_{i}$ is the number of times the letter $a_{i}$ occurs in $x$.
\end{notation}

\begin{theorem}[{\cite[Propositions 6-8]{LAST91}}]

    Let $\F_{d}$ be the free group of $d$ generators, and $\mu$ be the probability measure inducing the nearest neighbour lazy random walk on $\F_{d}$. Let also $(n_{j}, y_{j})_{j\in\N}$ be a sequence in $\ST_{e}$ such that $(n_{j}, y_{j})\to\infty$. Then, for $(m, x)\in\ST_{e}$ and $\xi(n_{j}, y_{j})\to\xi = (\xi_{1}, \xi_{2}, \dots, \xi_{d})$:
    \begin{enumerate}[(i)]
        \item if $y_{j} = y\in\F_{d}$ for large enough $j\in\N$, we get
        \begin{equation*}
            K_{ST}((R, y), (m, x)) := R^{m}H(x, y);  
        \end{equation*}
        \item if $y_{j}\to\omega\in\partial \F_{d}$ as $j\to\infty$ and $\sum^{d}_{i = 1}\xi_{i} < 1$, we get
        \begin{equation*}
            K_{ST}((\lambda, \omega), (m, x)) := \lambda^{m}K_{\lambda}(x, \omega),
        \end{equation*}
        where $\lambda = e^{s(\xi)}$ with $0 < \lambda \leq R$;
        \item if $y_{j}\to\eta\in\partial \F_{d}$ as $j\to\infty$ and $\sum^{d}_{i = 1}\xi_{i} = 1$, we get that the limit space-time Martin kernel is $0$ unless $|x| = m$ and it lies on the unique geodesic from $e$ to $\eta$, so
        \begin{equation*}
            K_{ST}((0, \eta), (m, x)) := (\mu(x_{1})\mu(x_{2})\dots\mu(x_{m}))^{-1}.
        \end{equation*}
    \end{enumerate}
\end{theorem}

\begin{proposition}

    Let $\F_{d}$ be the free group of $d$ generators, and $\mu$ be the probability measure inducing the nearest neighbour lazy random walk on $\F_{d}$. Then, for each $z\in\F_{d}$, the $\pi_{z}$ is a boundary representation.
\end{proposition}

\begin{proof}
    
    Similar to \cite[Proposition 3.11]{DOMA17}, we will use \cite[Theorem 7.2]{ARWI11} to show that each $\pi_{z}$ is completely strongly peaking in the sense of Definition (REF). By the above decomposition of representations, it suffices to show that there exists $T\in \mathcal{T^{+}}(\Gamma, \mu)$ such that $\|\pi_{z}(T)\| > \sup_{\pi_{\mathcal{V}}}\|\pi_{\mathcal{V}}(T)\|$, where supremum is take over all $\pi_{\mathcal{V}}$ that annihilate $\mathcal{I}(\Gamma, \mu)$, and $\|\pi_{z}(T)\| > \sup_{\pi_{\mathcal{I}}}\|\pi_{\mathcal{I}}(T)\|$, where supremum is take over all $\pi_{\mathcal{I}}$ is the unique extension of $\pi_{z}|_{\mathcal{I}(\Gamma, \mu)}$ to $\mathcal{T}(\Gamma, \mu)$. For the former inequality, note that the norm \dots For the latter, using (REF) and (REF), we reduce the problem to showing that $\|\pi_{z}(T)\| > \sup_{z'\neq z}\|\pi_{z'}(T)\|$. Since each $\pi_{z}$ is equivariant with respect to the action of $\Gamma$, we reduce the problem to showing the inequality just for $\pi_{e}$.

    Now, let $z\in\Gamma$ with $z\neq e$. For $m, n\in\N$ and $(x, y)\in\Gr(P^{m})$, define the operators
    \begin{equation*}
        \begin{split}
            T^{(n)}_{x, y}\colon \mathcal{H}(\Gamma, \mu) & \to \mathcal{H}(\Gamma, \mu), \\
            e^{(m)}_{y', z} & \mapsto \delta_{y, y'}\sqrt{\dfrac{P^{m}(y, z)}{P^{n + m}(x, z)}}e^{(n + m)}_{x, z},
        \end{split}
    \end{equation*}
    and choose $T := T^{(n)}_{e, e}$. These operators are bounded due to \dots, and their adjoints are given by
    \begin{equation*}
        \begin{split}
            (T^{(n)}_{x, y})^{\ast}\colon \mathcal{H}(\Gamma, \mu) & \to \mathcal{H}(\Gamma, \mu), \\
                e^{(n + m)}_{x', z} & \mapsto \delta_{x, x'}\sqrt{\dfrac{P^{m}(y, z)}{P^{n + m}(x, z)}}e^{(m)}_{y, z},
        \end{split}
    \end{equation*}
    Since $T^{(n)}_{x, y} = \sqrt{\dfrac{1}{P^{n}(x, y)}}S^{(n)}_{x, y}$, we have that $T\in\mathcal{T}^{+}(\Gamma, \mu)$. Moreover, having the following decomposition
    \begin{equation*}
        \bigoplus_{z\in\Gamma}\ell^{2}(\ST_{z}(\Gamma, \mu)) = \bigoplus_{z\in\Gamma}\bigoplus_{m\in\N}\ell^{2}(\ST^{(m)}_{z}(\Gamma, \mu)),
    \end{equation*}
    we notice that each $\mathcal{H}^{(m)}_{z}(\Gamma, \mu) := \ell^{2}(\ST^{(m)}_{z}(\Gamma, \mu))$ and even the linear spans of each $e^{m}_{e, z}$ are reducing for $T^{\ast}T$. Hence $T$ is diagonalizable and for a fixed $n\in\N$, we get
    \begin{equation*}
        \left\|\pi_{e}(T)\right\|^{2} \geq \|\pi_{e}((T^{(n)}_{e, e})^{\ast}(T^{(n)}_{e, e}))(e^{(0)}_{e, e})\| = \left\|\dfrac{1}{P^{n}(e, e)}e^{(0)}_{z, z}\right\| = \dfrac{1}{P^{n}(e, e)}
    \end{equation*}
    and
    \begin{equation*}
        \begin{split}
            \left\|\pi_{z}(T)\right\|^{2} & = \|\pi_{z}(TT^{\ast})\| = \left\|TT^{\ast}|_{\mathcal{H}_{z}}\right\| \\
            & = \sup_{m\in\N}\left\|(T^{(n)}_{e, e})(T^{(n)}_{e, e})^{\ast}(e^{(m)}_{e, z})\right\| = \sup_{m\in\N}\dfrac{P^{m - n}(e, z)}{P^{m}(e, z)}\left\|e^{(m)}_{e, z}\right\| \\
            & = \sup_{m\in\N}K_{ST_{e}}((e, n), (z, m)).
        \end{split}
    \end{equation*}
    Hence, we have to show the following
    \begin{equation*}
        \sup_{z\neq e}\left\|\pi_{z}(T)\right\|^{2} = \sup_{\substack{m\in\N \\ z\neq e}}K_{ST_{e}}((e, n), (z, m)) < \dfrac{1}{P^{n}(e, e)} \leq \left\|\pi_{e}(T)\right\|^{2}
    \end{equation*}
    For the supremum on the left hand side, we view the graph $\ST_{e}$ as a dense subset of the space-time Martin compactification $\Delta_{ST}\Gamma$. Now, its value can either be attained on $\ST_{e}$ itself, or on the space-time Martin boundary $\partial_{ST}\Gamma$. For the latter case, we obtain 
    \begin{equation*}
        \sup_{(z, m)\in\partial_{ST}\Gamma}K_{ST_{e}}((e, n), (z, m)) = \lim_{(z_{j}, m_{j})\to\infty}K_{ST_{e}}((e, n), (z_{j}, m_{j})).
    \end{equation*}
    Now, using Theorem 1.1, we break this down into three different cases.
    \begin{enumerate}[(i)]
        \item If $z_{j} = z\in\F_{d}$ for large enough $j\in\N$, then we have to show
            \begin{equation*}
                R^{n}H(e, z) = R^{n} < \dfrac{1}{P^{n}(e, e)}.
            \end{equation*}
        Using the local limit theorem for hyperbolic groups \cite[Theorem 1.1]{GOSE14}, we obtain
            \begin{equation*}
                P^{n}(e, e) \sim \beta(e, e)R^{-n}n^{-3/2}.
            \end{equation*}
        So, $P^{n}(e, e) < R^{-n}$ if we choose any $n > \beta(e, e)^{2/3}$.
        \item If $z_{j}\to\gamma\in\partial\F_{d}$ as $j\to\infty$ and $\sum^{d}_{i = 1}\xi_{i} < 1$, then we need to show
            \begin{equation*}
                \lambda^{n}K_{\lambda}(e, \gamma) = \lambda^{n} < \dfrac{1}{P^{n}(e, e)},
            \end{equation*}
        where $0 < \lambda \leq R$. Since $\lambda^{n} \leq R^{n}$, this case reduces to the first one.
        \item If $z_{j}\to\eta\in\partial\F_{d}$ as $j\to\infty$ and $\sum^{d}_{i = 1}\xi_{i} = 1$, then 
        the space-time Martin kernel is identically $0$, i.e. the supremum can not be attained on the boundary.
    \end{enumerate}

    Now, if the value of the supremum is attained on $\ST_{e}$, we get
    \begin{equation*}
        \sup_{(z, m)\in\ST_{e}}K_{ST_{e}}((e, n), (z, m)) = \max_{(z, m)\in\ST_{e}}K_{ST_{e}}((e, n), (z, m)),
    \end{equation*}
    i.e. there exists $(m_{0}, z_{0})\in\ST_{e}$ for which the value is attained. Now, using Proposition \ref{prop:probability:inequalities}, we find $n\in\N$ such that $0 < P^{n}(e, e) < 1$ and
    \begin{equation*}
        P^{n}(e, e)P^{m_{0}}(e, z_{0}) < P^{n + m_{0}}(e, z_{0}),
    \end{equation*}
    and so 
    \begin{equation*}
        K_{ST_{e}}((e, n), (z_{0}, m_{0})) = \dfrac{P^{m_{0} - n}(e, z_{0})}{P^{m_{0}}(e, z_{0})} < \dfrac{1}{P^{n}(e, e)},
    \end{equation*}
    which concludes the proof.
\end{proof}

\end{document}

